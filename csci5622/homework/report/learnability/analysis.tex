\documentclass[11pt]{article}
\usepackage[letterpaper,margin=1in]{geometry}
\usepackage{color}
\usepackage[dvipdfmx]{graphicx}
\usepackage{amsbsy}
\usepackage{amsmath}
\usepackage{adjustbox}
\usepackage{url}
\usepackage{mathtools}
\DeclarePairedDelimiter\ceil{\lceil}{\rceil}
\DeclarePairedDelimiter\floor{\lfloor}{\rfloor}

\newcommand{\argmax}{\mathop{\rm arg~max}\limits}

\begin{document}
\title{Analysis Report on Assignment 5: PAC Learnability}
\author{Yoshinari Fujinuma}
\date{}
\maketitle

\section{Problem 1}
% http://grandmaster.colorado.edu/~cketelsen/files/csci5622/videos/lesson12/lesson12.pdf
``any consistent learner using $H=C$ will, with probability $95\%$, output a hypothesis with error at most $0.15$''

error $\epsilon = 0.15$

confidence $\delta = 0.05$

What is the number of sufficient training data size $m$?

Since class $C$ is finite, the hypothesis $H$ is also finite.

The number of hypothesis $|H|$ is any combinations of three different points:
$$
|H| = \dbinom{N}{3} = 161700
$$
where $N$ is the number of points in $[0, 99]$. So $N = 100$.

According to Lecture 12\footnote{http://grandmaster.colorado.edu/~cketelsen/files/csci5622/videos/lesson12/lesson12.pdf}, the concept $c$ is PAC learnable with 
$$
m \geq \frac{1}{\epsilon}(\log |H| + \log(\frac{1}{\delta}))
$$
By plugging in $\epsilon = 0.15$ and $\delta = 0.05$, 
$$
m \geq \frac{1}{0.15}(\log(161700) + \log(\frac{1}{0.05})) \approx 99.928
$$
So the bound or the minimum number of training examples necessary is $m = \ceil{99.928} = 100$.


\section{Problem 2}
``State and prove the VC Dimension of the hypothesis class H of linear hyperplanes in 2D that pass through the origin.''

According to Lecture 13\footnote{http://grandmaster.colorado.edu/~cketelsen/files/csci5622/videos/lesson13/lesson13.pdf}, the VC dimension is defined as the following:
$$
\mbox{VCdim}(H) = max{|S|: H \mbox{ shatters } S \mbox{ for some } S}
$$

Decision boundary: $ +1 : \mbox{ if }y' > y = a * x'$

\subsection{Proof of the lower bound}

For a given $2$ points in 2D, we can shatter in similar way to slides 22 through 25.

\subsection{Proof of the upper bound}

\end{document}

