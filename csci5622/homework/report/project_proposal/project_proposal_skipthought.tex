\documentclass[11pt]{article}
\usepackage[letterpaper,margin=1in]{geometry}
\usepackage{color}
\usepackage[dvipdfmx]{graphicx}
\usepackage{amsbsy}
\usepackage{amsmath}
\usepackage{adjustbox}
\usepackage{url}
%\usepackage{floatrow}
\usepackage[font=small,labelfont=bf,tableposition=top]{caption}
\DeclareCaptionLabelFormat{tableonly}{\tablename~\thetable}
%\newfloatcommand{capbtabbox}{table}[][\FBwidth]


\newcommand{\argmax}{\mathop{\rm arg~max}\limits}

\begin{document}
\vspace{-1cm}
\title{\vspace{-2ex}Project proposal for the ML class final project\vspace{-2ex}}
\author{Yoshinari Fujinuma\vspace{-2ex}}
\date{\vspace{-2ex}}
\maketitle

\vspace{-0.5cm}

\section{Improving Tweet2vec by using skip-thought vectors}

One problem of tweets:

\begin{enumerate}
 \setlength\itemsep{0.01em}
 \item Tweets are short and a single tweet is usually not enough\footnote{Prof. Leysia Polen also mentioned it during last week's Intro. to CS PhD class}\footnote{Also mentioned in Jordan's hyperlocal EPIC grant: ``a challenge in topic modeling tweets is finding the right unit of analysis: individual tweets are often too small, but whole user streams can sometimes be too big.''}.
\end{enumerate}

One idea is to use Skip-thought vectors \cite{kiros2015skip} on user-aggregated tweets. The two RNN decoders produce one tweet before and one tweet after the target tweet.

{\bf Objective of the project}: Apply skip-thought vectors on user-aggregated tweets to embed tweets.

\begin{figure}[htb]
  \begin{center}
     \scalebox{0.5}
      {\includegraphics[]{pictures/skipthought.png}}

      \caption{An example of skip-thought vector \cite{kiros2015skip}. }
      \label{fig:learning_rate}
     \end{center}
\vspace{-0.5cm}
\end{figure}

\begin{figure}[htb]
  \begin{center}
     \scalebox{0.5}
      {\includegraphics[]{pictures/skipthought_results.png}}

      \caption{Example results of skip-thought vector \cite{kiros2015skip}. }
      \label{fig:learning_rate}
     \end{center}
\vspace{-0.5cm}
\end{figure}




\bibliographystyle{plain}
\bibliography{project_proposal_skipthought}

\end{document}
