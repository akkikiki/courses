\documentclass[11pt]{article}
\usepackage[letterpaper,margin=1in]{geometry}
\usepackage{color}
\usepackage[dvipdfmx]{graphicx}
\usepackage{amsbsy}
\usepackage{amsmath}
\usepackage{adjustbox}
\usepackage{url}

\newcommand{\argmax}{\mathop{\rm arg~max}\limits}

\begin{document}
\title{Analysis Report on Assignment 6: Feature Engineering}
\author{Yoshinari Fujinuma (Kaggle username: yofu1973)}
\date{}
\maketitle

\section{Feature Engineering on the L2 regularized Logistic Regression}

Strategy:
\begin{enumerate}
 \item Include both generalizable and meaningful feature 
 \item Avoid unmeaning feature to avoid confusing during the training
\end{enumerate}


%Held out dev dataset.
Out of the 14,000 training examples, the first 10,000 examples are used during the training, the rest of 4,000 examples are used as the held out development dataset.

Features:
\begin{enumerate}
 \item unigram
 \item bigram
 \item trope names
 \item word embeddings
 \item named entity 
 \item Genre of the movie (From IMDB\footnote{ftp://ftp.fu-berlin.de/pub/misc/movies/database/genres.list.gz})
\end{enumerate}

After including enough all features except for unigrams and throwing away the unigram feature, the accuracy of the prediction increased.

In the initial trial and errors, and looking into Jorndan's paper, some unigrmas such as ``kill'', ``death'' so even after we drop unigram features, we add them back to the mdoel.

\begin{table}[h]
  \centering
  \begin{tabular}{|l|l|r|r|r|r|}
  \hline \bf Feature     & \bf Held out Accuracy     \\ \hline
   All                   & 0.693 \\ \hline
  \end{tabular}
\end{table}

\section{Error Analysis}

Second episode.


%The following two graph shows the relationship between training examples and accuracy:
%\begin{figure}[htb]
%  \begin{center}
%   \begin{tabular}{c}
%
%    \begin{minipage}{0.5\hsize}
%     \begin{center}
%     \scalebox{0.33}
%      {\includegraphics[]{figure_1.png}}
%   
%      \caption{The relationship between the number of training examples and accuracy. The value of $k$ is fixed to $3$. }
%      \label{fig:corpus_size}
%     \end{center}
%    \end{minipage}
%
%    \begin{minipage}{0.01\hsize}
%    \end{minipage}
%
%    \begin{minipage}{0.5\hsize}
%     \begin{center}
%      \scalebox{0.33}
%      {\includegraphics[]{figure_2.png}}
%      \caption{\label{k_and_accuracy}The relationship between $k$ and accuracy. The value of the number of training examples is fixed to $500$}
%     \end{center}
%    \end{minipage}
%
%  \end{tabular}
% \end{center}
%\end{figure}

\end{document}

